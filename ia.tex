\documentclass{article}

\usepackage{todonotes}

\addtolength{\oddsidemargin}{-.875in}
\addtolength{\evensidemargin}{-.875in}
\addtolength{\textwidth}{1.75in}

\addtolength{\topmargin}{-.875in}
\addtolength{\textheight}{1.75in}

\bibliographystyle{plain}

\begin{document}
\title{An Introduction to The Lambda Calculus}
\author{Erik Boesen}
\maketitle

\begin{abstract}
\end{abstract}

\section{What is the Lambda Calculus?}
The lambda calculus (hereinafter notated as ``the $\lambda$ calculus'') is a system of mathematical notation and logic developed for the study of the elementary properties of functions \cite{rojastutorial}.

The notation and concept of the $\lambda$ calculus was first described in 1932 by Alonzo Church \cite{church}. It underwent revisions after being the subject of criticism that it was logically invalid \cite{church2}. Later, the system of notation gained new notoriety as the basis for new forms of computer programming.

The $\lambda$ calculus is sometimes referred to as a universal programming language \cite{rojastutorial}, given that it can represent and perform the computations of a ``turing machine'' (a computer) and vice versa. The $\lambda$ calculus is the basis for functional programming languages, or those whose attributes including treating functions as data and using a declarative programming paradigm \cite{hudakevolution}. The topic of functional programming is beyond the scope of this paper, however, interested readers are invited to investigate the topic through papers including \cite{totalfp} and \cite{hudakevolution}. Throughout this paper, the minimalist functional language Scheme will\todo{Is it?} be occasionally used for illustration of $\lambda$ calculus concepts. Any opaque logic will be explained.

\section{Notation}
The notation used to describe $\lambda$ calculus expressions seldom bears resemblance to more traditional systems of mathematical notation. To describe, for instance, the identity function (in which a single parameter is given as input and is outputted without modification) in traditional notation, one would likely use some variation upon:
$$f(x)=x$$
To make the function anonymous in traditional notation, one could use:
$$(x) \mapsto x$$
In the $\lambda$ calculus, however, an anonymous identity function would be notated thus:
$$\lambda x.x$$
Lambda expressions can be applied by parenthesizing them and appending parameters afterward. So, to apply our identity function to some constant $C$, we would perform the following lambda computation:
$$(\lambda x.x)C$$
$$=\lambda x.C$$
$$=C$$
Some explanation of the symbolic logic used herein is in order. Only two symbols in the calculus are defined, $\lambda$ and $.$ (some other symbols, such as parentheses, are used intuitively for grouping and other purposes). The former is used to represent a function definition, whereas the latter separates parameters from that function's internal logic. The sum of those two components is referred to as an ``abstraction.'' The $\lambda$ calculus also defines ``applications,'' or one lambda expression applied to another \cite{horowitz}.

The important thing to note about the aforementioned syntax is that in pure $\lambda$ calculus, there are no other symbols or gramatical constructions used. This leads to some occasionally verbose definitions for otherwise simple concepts.\todo{Add example of convoluted definition of some simple operation}

Throughout this paper, we will generally try to keep to the pure $\lambda$ calculus where possible, however, we may use some traditional mathematical symbols to illustrate.

The $\lambda$ calculus' syntax for representation of multi-parameter functions is potentially surprising. Purely speaking, there is no notion of multi-parameter functions. Rather, the $\lambda$ calculus makes use of ``currying,'' a technique (named after mathematician Haskell Curry) to transform an operation which would typically require multiple parameters into a series of functions requiring only one parameter each.
$$\lambda xy.x+y

\bibliography{research}
\end{document}
