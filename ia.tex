\documentclass{article}

\addtolength{\oddsidemargin}{-.875in}
\addtolength{\evensidemargin}{-.875in}
\addtolength{\textwidth}{1.75in}

\addtolength{\topmargin}{-.875in}
\addtolength{\textheight}{1.75in}

\bibliographystyle{plain}

\begin{document}
\title{An Introduction to The Lambda Calculus}
\author{Erik Boesen}
\maketitle

\begin{abstract}
\end{abstract}

\section{What is the Lambda Calculus?}
The lambda calculus (hereinafter notated as ``the $\lambda$ calculus'') is a system of mathematical notation and logic developed for the study of the elementary properties of functions \cite{rojastutorial}.

The notation and concept of the $\lambda$ calculus was first described in 1932 by Alonzo Church \cite{church}. It underwent revisions after being the subject of criticism that it was logically invalid \cite{church2}. Later, the system of notation gained new notoriety as the basis for new forms of computer programming.

The $\lambda$ calculus is sometimes referred to as a universal programming language \cite{rojastutorial}, given that it can represent and perform the computations of a ``turing machine'' (a computer) and vice versa. The $\lambda$ calculus is the basis for functional programming languages, or those whose attributes including treating functions as data and using a declarative programming paradigm \cite{hudakevolution}. The topic of functional programming is beyond the scope of this paper, however, interested readers are invited to investigate the topic through papers including \cite{totalfp} and \cite{hudakevolution}. Throughout this paper, the minimalist functional language Scheme will be occasionally used for illustration of $\lambda$ calculus concepts. Any opaque logic will be explained.

\bibliography{research}
\end{document}
