\documentclass[twocolumn,titlepage,12pt]{article}

\usepackage[color=yellow]{todonotes}
\usepackage{amsmath}
\usepackage[hidelinks]{hyperref}
\usepackage[margin=1in]{geometry}

% Avoid crushing TODO notes.
% I opened https://github.com/henrikmidtiby/todonotes/issues/29 about this problem.
\setlength{\marginparwidth}{2cm}

\bibliographystyle{plain}

\begin{document}
\title{Lambda Calculus and the Y Combinator:  Creating Functional Recursion in a Language Without It}
\author{Erik Boesen, Candidate gsg256}
\date{\parbox{\linewidth}{\centering%
  \today\endgraf\bigskip
  Ms. Jennifer Jayson \& Mr. William Snyder\endgraf\medskip
  IB High Level Math\endgraf\medskip
  George Mason High School}}
\maketitle
\begin{abstract}
We explain the basics of the lambda calculus, an alternate system of mathematical representation and logic created in the 1930s by Princeton mathematician Alonzo Church. We take advantage of this new intuition to explain the Y combinator, a simple yet powerful construct for simulating the process of recursion in a mathematical language lacking such a concept or even the ability to name functions.
\end{abstract}

\onecolumn
\tableofcontents
\newpage
\twocolumn

\section{What is the Lambda Calculus?}
Lambda calculus (interchangeably notated ``$\lambda$-calculus'' and with or without the definite article ``the'') is a system of mathematical notation and logic developed for the study of the elementary properties of functions \cite{rojastutorial}. The calculus defines a set of very simple rules and logical operations which, shockingly, can express \textit{any} mathematical operation.

\subsection{A Brief History}
The notation and concept of the lambda calculus was first described in the 1930s by Alonzo Church \cite{church}.\footnote{Church later published a revision \cite{church2} of the system after other researchers criticized it as logically inconsistent \cite{logicallyinconsistent}. This revision brought the calculus to roughly its current form.} Despite lacking many attributes of standard maths, including basic operators and even a native concept of numbers, the system can be used to represent any computation traditionally possible. Even complex computations such as differential calculus \cite{differentiallc}\footnote{That calculus learned by millions of students each year in high school classrooms.} can be represented using the lambda calculus. The lambda calculus is equivalent in capability to a ``Turing machine''---a mathematical notion of a computer.

Alan Turing, who proposed the idea of Turing machines, sought in parallel to Church to solve the \textit{Entscheidungsproblem}. Translating to ``decision problem'', this question was posed by logician David Hilbert in 1928, and asked whether an algorithm could exist to take as input a first-order logical construction, and output a boolean (true or false, yes or no) stating whether that logical construction is universally valid \cite{hilbert}. Both Church and Turing proved, independently and in different countries, that such an algorithm could not exist. Church used his construct of the lambda calculus to respond to the problem, while Turing used the competing---and comparatively simple---construct of the Turing machine.\footnote{Interestingly, despite their sometimes conflicting ideas on computation, Turing studied as a PhD candidate under Church from 1936 to 1938 \cite{churchpapers}.} The joint negatory answer given by the two computer scientists to the \textit{Entscheidungsproblem} is encompassed under the famous ``Church-Turing thesis'' \cite{churchturingthesis}. Unlike Turing machines, the lambda calculus never uses internal state to make decisions: functions cannot use stored data, and the same inputs will always yield the same outputs.

More recently, the lambda calculus and its logical process has gained new notoriety as the basis for new forms of computer programming. The lambda calculus is a ``universal programming language'' \cite{rojastutorial}, and can, on its own, represent and perform the computations of a Turing machine and vice versa, even without a computer to run it. The lambda calculus has thus become the basis for ``functional'' programming languages, or those whose attributes include treating functions as data and using a declarative programming paradigm\footnote{Declarative programming refers to the style of programming or mathematical notation where logic, not ``control flow,'' of a program is defined \cite{declarativeprogadv}. Rather than specifying a sequence of instructions, a declarative programmer would specify what logic functions contain, not necessarily in a linear manner. In short, while non-declarative (``imperative'') programming involves determination of what a function should \textit{do}, declarative programming emphasizes what a function \textit{is}. This structural dichotomy is reminiscent of that between Church and Turing's ideas on computation.} \cite{hudakevolution}. The topic of functional programming is beyond the scope of this paper. Interested readers, however, are invited to investigate the topic through papers including \cite{totalfp} and \cite{hudakevolution}.

\subsection{Motivations for Research}
I personally have long been fascinated by the ideas of the lambda calculus and by the way this incredible area of mathematics allows for any operation to be broken down into such simple base operations. I have worked in the past to learn and use computer programming languages such as Lisp, Scheme, and Haskell, all of which are based off the concepts of the lambda calculus. Before writing this paper, I had even researched the basic workings of the calculus, however, my understanding was mostly based off of a shallow reading through the topic's page on Wikipedia. When the assignment to perform in-depth research on a maths topic was announced, I knew immediately that I wanted to take a deep dive into the logical construct advertised by Wikipedia as having been introduced ``as part of [Alonzo Church's] research of the foundations of mathematics" \cite{wikipedialc}.

In this Internal Assessment, I will explore the lambda calculus and seek to advance my own understanding through creating for readers a simplistic yet sophisticated explanation of the lambda calculus and its applications. Students are better able to learn new material if they proceed through the process of educating others about that topic. Hence, I will set forward with the goal of creating instructional research within the field I've selected.

An important fact to recognize is that the lambda calculus is infrequently used in modern mathematics. Most areas of maths research use the notation and logical systems with which every mathematician from elementary school children and on are familiar. However, gaining an understanding of how this system works allows us a glimpse at how some types of computation in computer programming work and an understanding of the different goals and capabilities of competing forms of mathematics. Finally, understanding the lambda calculus offers a deeply reflective intellectual exercise: the understanding of how it works---and the fact \textit{that} it works despite ignoring traditional mathematical technique---was a humbling and educational experience for myself. I hope readers will be similarly incredulous that this total alternative to usual ways of doing math has evaded their attention for so long.

\section{Lambda Notation}
\subsection{Comparisons to Conventional Notation}
The notation used to describe lambda calculus expressions seldom bears intentional resemblance to more traditional systems of mathematical notation. To describe, for instance, the identity function (in which a single parameter is given as input and is then returned without modification) in traditional notation, one would likely use some variation upon:
$$f(x)=x$$
To make the function anonymous---remove its name---in traditional notation, one could use a ``map'' expression \cite{intrographtheory}, which simply lists the parameters to a function and, following the map symbol $\mapsto$, gives the function's internal logic:
$$x\mapsto x$$
Functions in traditional mathematics can easily be translated to use anonymous mapping. For example, the function for finding the length of a three-dimensional vector
$$L(x,y,z)=\sqrt{x^2+y^2+z^2}$$
could become
$$(x,y,z) \mapsto \sqrt{x^2+y^2+z^2}$$

\subsection{The Basics}
Understanding this syntax, the transition to lambda calculus notation is relatively intuitive. Let us return to our identity map $x\mapsto x$ from above. In the lambda calculus, an anonymous identity function would be notated thus:
$$\lambda x.x$$
In addition to variables, which, as in conventional mathematics, can be any symbol, only two operational symbols in the lambda calculus are defined. These are $\lambda$ and $.$ (some other symbols, such as parentheses, are used intuitively for grouping and other purposes). The former is used to represent a function definition, whereas the latter separates parameters from that function's internal logic. The sum of those two components is referred to as an ``abstraction,'' which is roughly a synonym for a function definition.

The lambda calculus also defines ``application,'' the passing of a  function to a parameter to a function \cite{horowitz}. Lambda calculus expressions associate left-to-right. So, lambda abstractions can be applied simply by parenthesizing the abstraction and appending a parameter afterward.

\subsection{Reduction Operations}
\subsubsection{$\beta$-reduction}
After one constructs a lambda expression, one can solve it through the process of ``reduction.'' The most prevalent form of reduction is the process of substituting a parameter into a lambda function. That process is referred to as $\beta$-reduction (beta-reduction) \cite{hudakintro}.

As an example of $\beta$-reduction, to apply our identity function to some other variable $C$, we would perform the following lambda computation:
$$(\lambda x.x)C$$
$$=x[x:=C]$$
$$=C$$
You will notice that we here make use of the syntax $x[x:=C]$. This is an idiom used during $\beta$-reduction to make clear the process of substitution being undertaken.

\subsubsection{$\alpha$-conversion}
A second reduction operation is $\alpha$-conversion (alpha-conversion), in which one variable name is substituted for another to prevent naming collisions, as in $\lambda x.B[x] \to \lambda y.B[y]$, where $B$ represents the lambda body \cite{hudakintro}.}

$\alpha$-conversion may be necessary in cases where names of variables clash. Take the following lambda expression:
$$\lambda y.\lambda x.yx$$
Alone, this expression raises no conflicts. However, imagine what would happen if we were to apply the expression to another with similar variable names:
$$(\lambda y.\lambda x.yx)(\lambda x.x)$$
We can first perform $\beta$-reduction as before.
$$\lambda x.yx[y:=\lambda x.x]$$
And here arises the problem. We are unable to complete the reduction, because doing so would cause the variable $x$ to be used in multiple ways.
$$\lambda x.(\lambda \textbf{x}.\textbf{x})x$$
The variables $\textbf{x}$ rendered above in bold are different from the others, but their names are unable to distinguish between the two. So, to reduce this expression properly and without conflict, we can use $\alpha$-conversion. Let us restart the process of reduction:
$$(\lambda y.\lambda x.yx)(\lambda x.x)$$
We perform $\alpha$-reduction first, using standard notation:
$$(\lambda y.\lambda x.yx)(\lambda x.x[z])$$
We are then left with:
$$(\lambda y.\lambda x.yx)(\lambda z.z)$$
We can now use $\beta$-reduction to simplify this expression without conflict.
$$\lambda x.yx[y:=\lambda z.z]$$
$$\lambda x.(\lambda z.z)x$$

\subsubsection{$\eta$-conversion}
The final reduction operation, arguably the most esoteric of the three, is $\eta$-conversion (eta-conversion), which states that for a bound variable x, $\lambda x.Bx \to B$. Stated more transparently, if there exists a lambda abstraction which does nothing beyond applying another abstraction to its single parameter, the outer abstraction can be removed in favor of a direct use of the wrapped one \cite{etared}.

As an example of how $\eta$-conversion works, let us take the expression with which finished in the above illustration of $\alpha$-conversion:
$$\lambda x.(\lambda z.z)x$$
This is, of course, a perfectly valid lambda abstraction. However, when we take a closer look at what it does when applied, we realize that it does nothing but apply another abstraction, $\lambda z.z$, to its one parameter. So, since through $\beta$-reduction,
$$(\lambda x.(\lambda z.z)x)C\to (\lambda z.z)x[x:=C]\to (\lambda z.z)C$$
We can therefore use $\eta$-conversion to transform $\lambda x.(\lambda z.z)x$ into simply $\lambda z.z$.

\subsection{Bound Variables}
A final important distinction in the lambda calculus is that between bound and free variables. A bound variable is one which is a parameter to the lambda abstraction, whereas a free variable is any other variable. So, in the expression
$$\lambda \theta.\theta \xi$$
$\theta$ is a bound variable, and $\xi$ a free variable because while $\theta$ is a parameter to the abstraction given, $\xi$ is not \cite{stanfordlc}.

\subsection{Currying}
The lambda calculus' syntax for representation of multi-parameter functions is potentially surprising. Purely speaking, there is no notion of multi-parameter functions. Rather, the lambda calculus makes use of ``currying,'' a technique named after mathematician Haskell Curry. In this process, a function which would typically require multiple parameters transforms into a series of functions requiring only one parameter each. So,
$$\lambda xyz.xyz$$
would expand to
$$\lambda x.\lambda y.\lambda z.xyz$$
These two expressions are equivalent, assuming that the impure syntax used in the first represents one function with three parameters $x$, $y$, and $z$. Let us see how this expression would be applied:
$$(\lambda x.\lambda y.\lambda z.xyz)ABC$$
The outer abstraction would be applied first, and the single parameter it accepts, $x$, would be substituted through $\beta$-reduction into the body of the enclosed abstraction. So, the expression would reduce as follows:
$$(\lambda y.\lambda z.xyz[x:=A])BC$$
$$(\lambda y.\lambda z.Ayz)BC$$
A similar process takes place with the second parameter, $B$:
$$(\lambda z.Ayz[y:=B])C$$
$$(\lambda z.ABz)C$$
Then, finally, $\beta$-substitution is used for a final time to yield the fully reduced expression:
$$ABz[z:=C]$$
$$ABC$$

\section{Illustration: Defining Exponentiation}
The important thing to note about the aforementioned syntax is that in pure lambda calculus, there are no other symbols or syntactic constructions used. This leads to some occasionally verbose definitions for otherwise simple concepts, which must be redefined before they can be used. One must go through a frankly absurd number of operations in order to perform the simplest of mathematical processes. To illustrate, let us walk through the process of defining exponentiation. To any remotely experienced mathematician the operation is doubtless well known. In the lambda calculus, however, we must do a great deal of work to define this capability.

\subsection{Church Numerals}
First, we must define what numbers are.\footnote{Yes, seriously.} A common way to use numbers in the lambda calculus is through ``Church numerals.'' Church numerals are actually themselves lambda abstractions, which take in a function and compose it $n$ times. Church numerals are notated with an underline \cite{cornelllc}.
$$\underline{0}=\lambda f.\lambda x.x$$
$$\underline{1}=\lambda f.\lambda x.fx$$
$$\underline{2}=\lambda f.\lambda x.f(fx)$$
$$\underline{3}=\lambda f.\lambda x.f(f(fx))$$
$$\underline{4}=\lambda f.\lambda x.f(f(f(fx)))$$
$$...$$

\subsection{Building Upward}
The first step, having now invented numbers, towards exponentiation is the ``successor'' function. This lambda expression takes a function $f$ as input, then composes it so as to return, for input $n$, $n+1$.
$$\lambda n.\lambda f.\lambda x.f((n f) x)$$

Building upon the successor function, we can define addition as follows:
$$\lambda m.\lambda n.\lambda f.\lambda x.(m f) ((n f) x)$$
Put simply, we execute succession $n$ times to add $n$ to the numeral $m$.

Building off this definition of addition, we can define multiplication as simply repeated addition. We'll need to use $\alpha$-conversion, renaming variables in the aforementioned addition function:
%$$\lambda m.\lambda n.\lambda fx.(m f) ((n f) x) \to \lambda m.\lambda n.\lambda fx.(m f) ((n f) x)$$
$$\lambda a.\lambda b.a ((\lambda m.\lambda n.\lambda fx.(m f) ((n f) x)) b) \underline{0}$$
This expression is clearly convoluted and difficult to interpret. This is a key flaw within the pure lambda calculus. Without the ability to name abstractions, simple expressions quickly become lengthy and inconvenient to deal with.\footnote{It should be noted that some sources, including \cite{cornelllc}\cite{rojastutorial} and others, as well as in most functional programming languages based upon the lambda calculus, mitigate this issue by naming lambda expressions and substituting in the expressions when solving. So, rather than the convoluted expression $\lambda a.\lambda b.a ((\lambda m.\lambda n.\lambda fx.(m f) ((n f) x)) b) \underline{0}$, one could define $ADD=\lambda m.\lambda n.\lambda fx.(m f) ((n f) x)$ and then write multiplication as $\lambda a.\lambda b.(ADD b) \underline{0}$. However, this is not pure lambda calculus.}

The relationship between addition and multiplication is roughly the same as that between multiplication and exponentiation. We simply need to repeat multiplication:
$$\lambda x.\lambda e.x ((\lambda a.\lambda b.a ((\lambda m.\lambda n.\lambda fx.(m f) ((n f) x)) b) \underline{0}) e) \underline{1}$$

In mathematical literature, occasionally conventional mathematical syntax will be mixed into lambda expressions. Strictly speaking, pure lambda calculus cannot use even the most basic of operators like $+$, $-$, $\times$, $\div$, exponents/roots, etc. However, for legibility and terseness, these symbols are often used. Reasonably so; it is far easier to understand this equivalent of the three-dimensional vector length formula:
$$\lambda x.\lambda y.\lambda z.\sqrt{x^2+y^2+z^2}$$
Conveying this operation as a pure lambda expression would require a prohibitively large quantity of space and writing.

\section{Implementing Recursion in the Lambda Calculus}
\subsection{What is Recursion?}
As has been previously alluded to, one of the most bizarre characteristics of the lambda calculus is its lack of natural support for naming functions. A consequence of this deficit is that the process of ``recursion'' is not as easy as one might be accustomed to.

Recursion is a technique used frequently within the field of computer science, but it finds much use within mathematical operations as well. A classic example of recursion is a redefinition of the factorial operator:
\[
factorial(x)=
\begin{cases}
    x\times factorial(x-1) & x>0 \\
    1 & x=0
\end{cases}
\]
This function may appear initially confusing. An illustration may help. Let us take the factorial of the integer $4$. First, we begin with our initial call to the function:
$$factorial(4)$$
As $4>1$, the first case is met, so:
$$factorial(4)=4\times factorial(3)$$
From here, we can clearly see how the expression will expand.
$$factorial(4)$$
$$=4\times factorial(3)$$
$$=4\times 3\times factorial(2)$$
$$=4\times 3\times 2\times 1\times factorial(0)$$
Since evaluating $factorial(0)$ uses the special ``base case'' of $factorial$, the expression will evaluate to:
$$4\times 3\times 2\times 1\times 1=24$$

Knowing, however, that the lambda calculus does not allow us to name abstractions, we must use an alternative construction known as the ``Y combinator'' to achieve our end in defining recursion.

\subsection{The Y Combinator}
Since lambda expressions are anonymous, no strategy for recursion using the notation is immediately evident. However, it is actually possible to create recursion even without naming any functions. To do so, we must implement the Y combinator \cite{ycombmedium}. The Y combinator originated in the lambda calculus as a useful tool for simulating recursion even in a calculus lacking names or otherwise not allowing for recursive computation. This technique was discovered by mathematician Haskell Curry, whom you may remember as the namesake of the process of currying \cite{compphileyc}.

To understand how the Y combinator works, we need to recall how lambda expressions associate. Lambda expressions are ``left-associative,'' meaning that the leftmost abstraction in an application will be evaluated first, and expressions to the right of one expression will be used as parameters to that on the left. \textit{Any} expression can be substituted as a parameter to a function. So:
$$(\lambda x.xx)\lambda x.x$$
$$=(\lambda x.x)(\lambda x.x)$$
$$=\lambda x.x$$

The Y combinator may be expressed as follows:
$$\lambda f.(\lambda x.f(x x))(\lambda x.f(x x))$$
To demonstrate how the Y combinator creates recursion, take the lambda abstraction $\lambda \omega.\omega$ (which, through $\alpha$-conversion, is equivalent to our identity function $\lambda x.x$ from earlier).

We start by applying the Y combinator, as above notated, to our abstraction.
$$(\lambda f.(\lambda x.f(x x))(\lambda x.f(x x)))(\lambda \omega.\omega)$$
In this instance, the Y combinator $(\lambda f.(\lambda x.f(x x))(\lambda x.f(x x)))$ accepts a single parameter of our lambda abstraction $\lambda \omega.\omega$ as parameter $f$. We can use $\beta$-reduction to reduce the expression by substituting $\lambda \omega.\omega$ for $f$ where it occurs within the Y combinator.
$$(\lambda x.f(x x))(\lambda x.f(x x))[f:=(\lambda \omega.\omega)]$$
$$(\lambda x.(\lambda \omega.\omega)(x x))(\lambda x.(\lambda \omega.\omega)(x x))$$
At this point we need to substitute the second outer lambda abstraction (not our identity) as the parameter $x$ to the first.\footnote{It is not necessarily to perform $\alpha$-conversion in this instance because the substitution of our new $x$ replaces all instances of that variable in the expression being applied. Therefore, there is no naming collision.}
$$(\lambda \omega.\omega)(x x)[x:=(\lambda x.(\lambda \omega.\omega)(x x))]$$
$$(\lambda \omega.\omega)((\lambda x.(\lambda \omega.\omega)(x x))(\lambda x.(\lambda \omega.\omega)(x x)))$$
At this point, you will notice that we are left roughly where we were before our second $\beta$-reduction, the only difference being that our reductions have left us with a copy of our identity abstraction, which we passed into the Y combinator as parameter $f$.

Through this process, the Y combinator is able to achieve recursion in the lambda calculus---no naming required.

To further understand the Y combinator, a fascinating exercise for the reader would be to reimplement the $factorial$ function previously described. Doing so requires implementation of boolean values (true and false), conditional operators, and subtraction. These topics are described extensively in a variety of sources, including \cite{stanfordlc}\cite{cornelllc}.

\section{Conclusion}
Through this paper we reviewed the basics of the lambda calculus, a new system of mathematical notation and logic. To summarize what we covered:
\begin{itemize}
  \item{The lambda calculus includes three syntactic rules for constructing ``lambda expressions.'' We can define variables, use ``abstractions'' to define lambda functions, and use ``application'' to apply those functions to parameters.}
  \item{We can solve lambda expressions through reduction operations, especially $\beta$-reduction, or substitution of parameters into the body of a lambda expression.}
  \item{We can use these simple rules to define any mathematical operation which can be run by a computer.}
  \item{We can implement recursion through a lambda expression known as the Y combinator.}
\end{itemize}
There are quite a few other fascinating techniques encompassed within the area of lambda calculus which are worthy of study. We did not have time to touch on the fascinating topic of boolean logic, which is a fascinating concept crucial in the construction of higher-level abstractions around the lambda calculus. Nor did we discuss the formulation of a type system, a topic crucial to using lambda calculus in the context of functional programming.

The lambda calculus uses a varied system of notation which sometimes involves, as previously mentioned, names for functions. In an effort to keep notation simple for reader understanding, we did not delve into this topic. There is, however, a great deal of literature regarding consistent names for functions which may be used to build larger and more complicated expressions \cite{stanfordlc}. The same is done for functional programming, which requires naming to abstract confusing logic so that, despite a basis in the lambda calculus, functional programming does not require excessive logic.

We also did not detail the topic of computability, despite this being the problem the calculus was designed to solve. The aforementioned Church-Turing thesis is a fascinating topic worthy of deep study, as well as Church and Turing's work on the infamous halting problem.

\section{Reflection}
I first chose to research the lambda calculus due to my existing interest in the field of computer science. I've had past experience with the functional programming paradigm, the study of which was a truly transformative experience. Conventional programming---with languages like C, Java, Python, etc.---is structured around traditional mathematical structures. When I learned about functional programming, I occasionally heard references to the lambda calculus, an apparently mysterious and confusing alternate form for mathematical syntax and logic.

Through this Internal Assessment, I had an excellent opportunity to dive into the new and fascinating notion of this calculus, and the incredible idea that any computation can be represented using only two predefined symbols.

I was at times challenged in comprehending the bizarre manner in which the calculus represents operations. It was tempting to impulsively use conventional syntax and operations like addition to build expressions! I eventually, though, figured out how to define exponentiation in terms of multiplication, multiplication in terms of addition, addition in terms of the successor function, all requiring knowledge of an esoteric way to define the concept of Church numerals.

It has been often said that learning functional programming makes one a better programmer. If this is so, I believe that learning and understanding the lambda calculus certainly makes one a better mathematician. I intend to continue researching the lambda calculus and to find new ways to apply it in my work within the fields of Computer Science and mathematics.

\newpage

\bibliography{research}
\end{document}
